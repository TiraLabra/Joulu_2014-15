\documentclass{article}
\usepackage[finnish]{babel}
\usepackage[utf8]{inputenc}
\author{Jaakko Hannikainen}

\title{Tietorakenteiden ja algoritmien harjoitustyö - Toteutusdokumentti}
\begin{document}
\maketitle

\section{Ohjelman yleisrakenne}
(main.cpp) main() alustaa signaalikäsittelijän, generoi tyhjää aluetta
aloituspaikan ympärille ja pyörittää input-render -silmukkaa. (ui.cpp) input()
lukee syötteet ja liikuttaa pelaajaa/kameraa niiden mukaisesti, ja lisää ja
poistaa palikoita. render() piirtää ensin lattian ja katon, joiden jälkeen se
piirtää jokaisen palikan tietyn etäisyyden päässä pelaajasta. Jos jossakin
kohtaa ei ole asetettu palikkaa tyhjäksi tai täydeksi, luodaan sinne uutta
aluetta. (generator.cpp) generate() ensin määrittää tietyn kokoisen alueen
kaivettavaksi, jonka jälkeen se antaa alueen satunnaiselle generaattorille,
(generators/*.cpp) joka puolestaan luo omanlaisensa alueen generate()n antaman
alueen päälle.

\section{Saavutetut aika- ja tilavaativuudet}
Quadtree on suurimaksi osaksi $O(\log_4 n)$, sillä se on yksinkertainen
nelihaarainen puu. Poikkeus on tosin map-metodi, joka on $O(n)$, koska se iteroi
puun läpi. Prim-generaattori on $O(|V|^2)$, life-generaattori on $O(|V|)$ ja
empty-generaattori on $O(|V|)$. Piirtosilmukka on $O(|V|^2)$, sillä siinä
käydään läpi $|V|$ alkiota, joille kutsutaan lineaarisesti kasvavaa
piirtofunktiota.

\subsection{Työn mahdoliset puutteet ja parannusehdotukset}
Työ on pelimäinen, mutta siinä ei ole paljoa pelillistämisominaisuuksia, kuten
esineitä ja vihollisia. Näitä voisi lisätä. Myös ylös- ja alaspäin voisi luoda
maailmaa, ja muokata piirtoa vastaavasti. Myös moninpeli voisi olla hyvä.

Piirto on kallista, sillä se ei käytä kaikkia OpenGL-ominaisuuksia, jotka
nopeuttaisivat piirtoa. Myöskin piilossa olevat ruudut piirretään, joka
aiheuttaa raskaamman piirtokutsun.

\end{document}
