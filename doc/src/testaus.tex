\documentclass{article}
\usepackage[finnish]{babel}
\usepackage[utf8]{inputenc}
\usepackage{geometry}
\author{Jaakko Hannikainen}

\title{Tietorakenteiden ja algoritmien harjoitustyö - Testausdokumentti}
\begin{document}
\maketitle

\section{Testatut tietorakenteet}
\begin{itemize}
    \item Qtree - quadtree
    \item QtreeNode - quadtreen noodi eli solmu tai lehti
    \item List - dynaaminen lista
    \item Queue - jono
\end{itemize}
Testaus on toteutettu siten, että jokainen testatun luokan rivi on suoritettu
testien ajon jälkeen. Lisäksi Qtree-luokkaa on testattu tehokkuuden puolesta,
ajaen add, contains, get ja map-metodeita suurilla syötteillä.

\section{Testauksen toistaminen}
\subsection{Virhetestaus}
\begin{enumerate}
    \item Käännä testit \\
        \texttt{\$ make test -j4} 
    \item Aja testit \\
        \texttt{\$ ./libtest}
\end{enumerate}
Testit tulostavat SUCCESS jokaisen onnistuneen testin vieressä, ja FAILURE
jokaisen epäonnistuneen testin vieressä. Testaus tulostaa lopuksi koosteen
testien onnistumisesta.

\subsection{Tehotestaus}
Tehotestauksen toistaminen vaatii gnuplot-nimisen ohjelman olemassaolon. Testien
suorittaminen kestää noin puoli minuuttia.
\begin{enumerate}
    \item Käännä ja suorita testit \\
        \texttt{\$ cd test/performance; make }
\end{enumerate}
Tehotestaus luo png-muotoisia graafeja testauksesta, jotka löytyvät samasta
kansiosta.

\end{document}
