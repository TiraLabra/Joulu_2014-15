\documentclass{article}
\usepackage[finnish]{babel}
\usepackage[utf8]{inputenc}
\usepackage{color, colortbl}
\definecolor{Gray}{gray}{0.9}
\author{Jaakko Hannikainen}

\title{Tietorakenteiden ja algoritmien harjoitustyö - Määrittelydokumentti}
\begin{document}
\maketitle

\section{Ratkaistava ongelma}
Tehokas dynaaminen luolaston luominen erilaisilla algoritmeilla, kun luolaston
koko lähenee ääretöntä.

\section{Toteutettavat tietorakenteet ja algoritmit}
Tietorakenteet:

\begin{table}[h]
\begin{tabular}{| l | l |}
\hline
\rowcolor{Gray}
Tietorakenne               & Mihin tarkoitukseen      \\ \hline
Quadtree                   & Luolaston säilyttämiseen \\ \hline
Dynaaminen lista           & Primin prioriteettijono  \\ \hline
\end{tabular}
\end{table}

\noindent
Luolaston luomiseen tarkoitetut algoritmit:
\begin{table}[h]
\begin{tabular}{| l | p{6.5cm} | l |}
\hline
\rowcolor{Gray}
Algoritmi   & Kuvaus                                & Ominaisuudet              \\ \hline
Prim        & Ota jonosta satunnainen piste, lisää  & Puumainen, helppo löytää  \\
            & jonoon nykyisen pisteen naapurit      & alkupiste                 \\ \hline
Kruskal     & Lisää kaikki pisteet joukkoon, ota    & Kaikki sokkelot yhtä      \\
            & satunnainen piste joukosta            & todennäköisiä             \\ \hline
Recursive   & Ota huone, jaa neljään osaan, leikkaa & Suorakulmiomaisia         \\ 
division    & kolmeen seinään neljästä aukko        & sokkeloita                \\ \hline
Random room & Sijoita suorakulmioita kartalle,      & Perinteinen roguelike-    \\ 
placement   & muodosta polkuja huoneiden välille    & generaattori              \\ \hline
Cellular    & Satunnainen alkutila, aja muutama     & Luonnollisen näköisiä     \\ 
automata    & generaatio B3/S1234-automataa         & luolia                    \\ \hline
Depth-first & Ota päälimmäinen piste pinosta, lisää & Pitkiä käytäviä           \\ 
            & naapurit satunnaisessa järjestyksessä &                           \\ \hline
\end{tabular}
\end{table}

\section{Tavoiteaika- ja tilavaatimus}
Aikavaatimustavoite $O(n^2)$, tilavaatimustavoite $O(n \log n)$.

\end{document}
