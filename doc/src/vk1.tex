\documentclass{article}
\usepackage[finnish]{babel}
\usepackage[utf8]{inputenc}
\usepackage{amsmath}
\usepackage{amsfonts}
\usepackage{color, colortbl}
\definecolor{Gray}{gray}{0.9}
\author{Jaakko Hannikainen}

\title{Tietorakenteiden ja algoritmien harjoitustyö - Viikkoraportti 1}
\begin{document}
\maketitle

\section{Mitä opin tällä viikolla}
\begin{itemize}
    \item C++:n syntaksin
    \item C++:n implisiittinen kopiokonstruktori ja assignment operator ovat
        rasittavia jos luokassa on käytetty osoittimia
    \item C++:n templatet ovat syntaksisokeria makrojen päällä, ja sen takia
        koodi, mikä yleensä kuuluisi .cpp-tiedostoihin joutuu .hpp-tiedostoihin
    \item Osaan implementoida quadtreen
    \item Quadtreen debuggaus on hankalaa
    \item Valgrind laskee muistivuodoksi sen, että allokoitua muistia ei vapauta
        ohjelman lopussa
\end{itemize}

\section{Mitä jäi epäselväksi}
\begin{itemize}
    \item ?
\end{itemize}

\section{Miten ohjelma on edistynyt?}
\begin{itemize}
    \item Quadtree toimii
    \item Voin luoda ennalta määrätyn kokoisen sokkelon.
    \item Ei muistivuotoja
\end{itemize}

\section{Mitä teen seuraavaksi?}
\begin{itemize}
    \item Dynaaminen luonti, kun pelaaja liikkuu
    \item Lisää luontialgoritmeja
\end{itemize}

\end{document}
